% Table of contents formatting
\renewcommand{\contentsname}{Table of Contents}
\setcounter{tocdepth}{4}

% Headers and page numbering
\usepackage{fancyhdr}
\pagestyle{plain}

% Following package is used to add background image to front page
\usepackage{wallpaper}

% Table package
\usepackage{ctable}% http://ctan.org/pkg/ctable

% Deal with 'LaTeX Error: Too many unprocessed floats.'
\usepackage{morefloats}
% or use \extrafloats{100}
% add some \clearpage

% % Chapter header
\usepackage{titlesec, blindtext, color}
\definecolor{gray75}{gray}{0.75}
\newcommand{\hsp}{\hspace{20pt}}
\titleformat{\chapter}[hang]{\Huge\bfseries}{\thechapter\hsp\textcolor{gray75}{|}\hsp}{0pt}{\Huge\bfseries}

% % Fonts and typesetting
\usepackage{fontspec}
\usepackage{libertine}

% FONTS
\usepackage{xunicode}
\usepackage{xltxtra}
\defaultfontfeatures{Mapping=tex-text} % converts LaTeX specials (``quotes'' --- dashes etc.) to unicode
\setmainfont[Scale=1.1]{Libertinus Serif}
\setsansfont[Scale=1.1]{Libertinus Sans}
\setmonofont[Scale=0.9]{Libertinus Mono}
\setmathfont[Scale=1.1]{Libertinus Math}
% \setromanfont[Scale=1.01,Ligatures={Common},Numbers={OldStyle}]{Palatino}
% \setromanfont[Scale=1.01,Ligatures={Common},Numbers={OldStyle}]{Adobe Caslon Pro}
%Following line controls size of code chunks
% \setmonofont[Scale=0.9]{Monaco}
%Following line controls size of figure legends
% \setsansfont[Scale=1.2]{Optima Regular}

% CODE BLOCKS
\usepackage[utf8]{inputenc}
\usepackage{listings}
\usepackage{color}
\usepackage{longtable}

% JAVA CODE BLOCKS
\definecolor{backcolour}{RGB}{255,255,255}
\definecolor{javapurple}{RGB}{143,	188,	187}
\definecolor{javagreen}{RGB}{94, 129, 172}
\definecolor{javared}{RGB}{94, 129, 172}
\definecolor{javadocblue}{rgb}{0.25,0.35,0.75}

\lstdefinestyle{pythonCodeStyle}{
  float=h,
  floatplacement=tbp,
  language=python,                         % the language of the code
  backgroundcolor=\color{backcolour},    % choose the background color; you must add \usepackage{color} or \usepackage{xcolor}
  basicstyle=\fontsize{7}{7},
  breakatwhitespace=true,
  breaklines=true,
  %breakindent=.5\textwidth,
  keywordstyle=\color{javapurple},
  stringstyle=\color{javared},
  commentstyle=\color{javagreen},
  %morecomment=[s][\color{javadocblue}]{/**}{*/},
  captionpos=t,                          % sets the caption-position to bottom
  frame=single,                          % adds a frame around the code
  numbers=left,
  numbersep=7pt,                         % margin between number and code block
  keepspaces=true,                       % keeps spaces in text, useful for keeping indentation of code (possibly needs columns=flexible)
  columns=fullflexible,
  showspaces=false,                      % show spaces everywhere adding particular underscores; it overrides 'showstringspaces'
  showstringspaces=false,                % underline spaces within strings only
  showtabs=false,                        % show tabs within strings adding particular underscores
  tabsize=1                              % sets default tabsize to 2 spaces
}
%breakindent=.5\textwidth,frame=single,breaklines=true%
%Attempt to set math size
%First size must match the text size in the document or command will not work
%\DeclareMathSizes{display size}{text size}{script size}{scriptscript size}.
%\DeclareMathSizes{12}{13}{7}{7}

% ---- CUSTOM AMPERSAND
% \newcommand{\amper}{{\fontspec[Scale=.95]{Adobe Caslon Pro}\selectfont\itshape\&}}

% HEADINGS
\usepackage{sectsty}
\usepackage[normalem]{ulem}
\sectionfont{\rmfamily\mdseries\Large}
\subsectionfont{\rmfamily\mdseries\scshape\large}
\subsubsectionfont{\rmfamily\bfseries\upshape\large}
% \sectionfont{\rmfamily\mdseries\Large}
% \subsectionfont{\rmfamily\mdseries\scshape\normalsize}
% \subsubsectionfont{\rmfamily\bfseries\upshape\normalsize}

% Set figure legends and captions to be smaller sized sans serif font
\usepackage[font={footnotesize}]{caption}

\usepackage{siunitx}

% Adjust spacing between lines to 1.5
\usepackage{setspace}
%\onehalfspacing
% \doublespacing
\raggedbottom

% Set margins
\usepackage[top=1.5in,bottom=1.5in,left=1.5in,right=1.4in]{geometry}
\setlength\parindent{0.5in} % indent at start of paragraphs (set to 0.3?)
\setlength{\parskip}{9pt}
\usepackage{indentfirst}

% Add space between pararaphs
% http://texblog.org/2012/11/07/correctly-typesetting-paragraphs-in-latex/
% \usepackage{parskip}
% \setlength{\parskip}{\baselineskip}

% Set colour of links to black so that they don't show up when printed
\usepackage{hyperref}
%\hypersetup{colorlinks=false, linkcolor=black}

% Tables
\usepackage{booktabs}
\usepackage{threeparttable}
\usepackage{array}
\usepackage{makecell}
\newcolumntype{x}[1]{%
>{\centering\arraybackslash}m{#1}}%
\usepackage{bookmark} % lets try this for weird tables
\usepackage{multirow} % cool tables and stuff of multiple rows.

% Allow for long captions and float captions on opposite page of figures
% \usepackage[rightFloats, CaptionBefore]{fltpage}

% Don't let floats cross subsections
% \usepackage[section,subsection]{extraplaceins}

% Rotate images and tables
\usepackage{float}
\usepackage{pdfpages}
\usepackage{pdflscape}
\usepackage{graphicx}
\usepackage{subfig}
\usepackage{rotating}
\usepackage{wrapfig}
\usepackage[export]{adjustbox}[2011/08/13]
\usepackage{floatpag}

\newenvironment{CSLReferences}[2] % #1 hanging-indent, #2 entry spacing
{\par\normalsize\normalfont
  \setlength{\parindent}{0pt}
  \setlength{\parskip}{\baselineskip}
  \setlength{\hangindent}{2em}
  \ignorespaces}
{\par}

% \newcommand{\CSLLeftMargin}[1]{}
% \newcommand{\CSLRightInline}[1]{}
% \newcommand{\CSLRightSkip}[1]{}

